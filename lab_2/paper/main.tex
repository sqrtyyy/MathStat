\documentclass[a4paper]{article}
\usepackage{amsmath}
\addtolength{\hoffset}{-2.25cm}
\addtolength{\textwidth}{4.5cm}
\addtolength{\voffset}{-3.25cm}
\addtolength{\textheight}{5cm}
\setlength{\parindent}{15pt}

\usepackage[unicode=true, colorlinks=false, hidelinks]{hyperref}
\usepackage[utf8]{inputenc}
\usepackage[english, russian]{babel}
\usepackage{mathtext}
\usepackage[T2A, TS1]{fontenc}
\usepackage{microtype} % Slightly tweak font spacing for aesthetics
\usepackage{amsthm, amssymb, amsmath, amsfonts, nccmath}
\usepackage{nicefrac}
\usepackage{epstopdf}
\usepackage[export]{adjustbox}
\usepackage{float} % Improved interface for floating objects
\usepackage{graphicx, multicol} % Enhanced support for graphics
\usepackage{pdfrender,xcolor}
\usepackage{breqn}
\usepackage{mathtools}
\usepackage{titling}
\usepackage{bm}
\usepackage{centernot}
\usepackage[cal=boondoxo,calscaled=.96]{mathalpha}
\usepackage{marvosym, wasysym} % More symbols
\usepackage{rotating} % Rotation tools
\usepackage{censor} % Facilities for controlling restricted text

\usepackage{fancyhdr}
\pagestyle{fancy}
\fancyhead{}\renewcommand{\headrulewidth}{0pt}
\fancyfoot[L]{}
\fancyhead{}
\fancyfoot{}
\fancyfoot[R]{\thepage}
\begin{document}
    \begin{titlepage}
   \begin{center}
       \vspace*{3cm}
       \large{САНКТ-ПЕТЕРБУРГСКИЙ ПОЛИТЕХНИЧЕСКИЙ УНИВЕРСИТЕТ}
       \vspace{0.4 cm}

       \large\textbf{Институт прикладной математики и механики}
       \vspace{0.4 cm}

       \large{Высшая школа прикладной математики и вычислительной физики}

       \vspace{3 cm}
       \normalsize\textbf{Отчет\\ по лабораторной работе №7\\ по дисциплине\\
«Математическая статистика»}
       \vfill
       \begin{flushright}
            \normalsize{Выполнил студент:\\
            Антонов Алексей\\
            группа: 3630102/80201}
            \vskip\medskipamount
            \normalsize{Проверил:

            к.ф.-м.н., доцент\\
            Баженов Александр Николаевич
            }
       \end{flushright}

       \vspace{0.8cm}


       \normalsize{Санкт-Петербург\\2021 г.}

   \end{center}
\end{titlepage}
    \tableofcontents
    \newpage
    \listoftables
    \newpage
    \section{Постановка задачи}
    Сгенерировать выборки размером 10, 100 и 1000 элементов.
    Для каждой выборки вычислить следующие статистические характеристики положения данных: $\overline{x}, med\,x, z_R, z_Q, z_{tr}$.
    Повторить такие вычисления 1000 раз для каждой выборки и найти среднее характеристик положения и их квадратов:
    \begin{equation}\label{eq:mean_formula}
        E(z)=\overline{z}
    \end{equation}
    Вычислить оценку дисперсии по формуле:
    \begin{equation}\label{eq:variance_formula}
        D(z)=\overline{z^2}-\overline{z}^2
    \end{equation}
    Представить полученные данные в виде таблиц.
    \section{Теория}
    \subsection{Вариационный ряд}
Последовательность $\displaystyle\{x_{(k)}\}_{k=1}^n$ элементов выборки размера $n$, расположенных в неубывающем порядке, называется вариационным рядом.
\subsection{Выборочные числовые характеристики}
\subsubsection{Характеристики положения}
\begin{itemize}
    \item Выборочное среднее
    \begin{equation}\label{mean}
        \overline{x}=\frac{1}{n}\sum_{i=1}^n x_i
    \end{equation}
    \item Выборочная медиана
    \begin{equation}\label{med}
        med\,x = \begin{cases}
        \displaystyle\;\;\;\;\;x_{(l+1)}&\text{при}\;\;n=2l+1\\
        \displaystyle\frac{x_{(l)}+x_{(l+1)}}{2}&\text{при}\;\;n=2l
        \end{cases}
    \end{equation}
    \item Полусумма экстремальных выборочных элементов
    \begin{equation}\label{exhfsum}
        z_R=\frac{x_{(1)}+x_{(n)}}{2}
    \end{equation}
    \item Полусумма квартилей\\
    Выборочный квартиль $z_p$ порядка $p$ определяется формулой
    \begin{equation}
        z_p = \begin{cases}\label{pqv}
        \displaystyle\;\;x_{([np]+1)}&\text{при}\;\;np\;\text{дробном,}\\
        \displaystyle\;\;\;\;\;x_{(np)}&\text{при}\;\;np\;\text{целом}
        \end{cases}
    \end{equation}
    Полусумма квартилей
    \begin{equation}\label{eq:hfsum}
        z_Q=\frac{z_{1/4}+z_{3/4}}{2}
    \end{equation}
    \item Усечённое среднее
    \begin{equation}\label{eq:trmean}
        z_{tr}=\frac{1}{n-2r}\sum_{i=r+1}^{n-r}x_{(i)},\;\;r\approx\frac{n}{4}
    \end{equation}
\end{itemize}
    \subsubsection{Характеристики рассеивания}
Выборочная дисперсия
\begin{equation}\label{eq:svar}
    D=\frac{1}{n}\sum_{i=1}^n \left(x_i-\overline{x}\right)^2
\end{equation}
    \section{Реализация}
        Лабораторная работа выполнена на языке Python в среде PyCharm с использованием следующих библиотек:
        \begin{enumerate}
            \item numpy
        \end{enumerate}
    \section{Результаты}
        \begin{table}[H]
            \centering
            \begin{tabular}{|c|c|c|c|c|c|}
                \hline
                 &$\overline{x}$&$med\ x$&$z_R$&$z_Q$&$z_{tr}$ \\ \hline
$E\left(z\right)$&0.0125&0.0038&0.0264&0.3263&0.2863\\ \hline
$D\left(z\right)$&0.0961&0.1256&0.1905&0.1166&0.1049\\ \hline
$E + \sqrt{D}$&0.3224&0.3582&0.4629&0.6678&0.6101\\ \hline
$E - \sqrt{D}$&-0.2975&-0.3506&-0.4101&-0.0152&-0.0375\\ \hline
\widehat{E}(z)&-&0.12&0.1&-&-\\ \hline

            \end{tabular}
            \caption{Нормальное распределение 10 элементов}
            \label{tab:norm_10}
        \end{table}

        \begin{table}[H]
            \centering
            \begin{tabular}{|c|c|c|c|c|c|}
                \hline
                 &$\overline{x}$&$med\ x$&$z_R$&$z_Q$&$z_{tr}$ \\ \hline
$E\left(z\right)$&-0.0002&0.0014&-0.0073&0.0168&0.0144\\ \hline
$D\left(z\right)$&0.0099&0.0156&0.0995&0.0122&0.0115\\ \hline

            \end{tabular}
            \caption{Нормальное распределение 100 элементов}
            \label{tab:norm_100}
        \end{table}

         \begin{table}[H]
            \centering
            \begin{tabular}{|c|c|c|c|c|c|}
                \hline
                 &$\overline{x}$&$med\ x$&$z_R$&$z_Q$&$z_{tr}$ \\ \hline
$E\left(z\right)$&-0.0003&-0.0008&-0.0041&0.0009&0.0022\\ \hline
$D\left(z\right)$&0.001&0.0016&0.0593&0.0012&0.0012\\ \hline
$E + \sqrt{D}$&0.0306&0.0388&0.2394&0.0361&0.0366\\ \hline
$E - \sqrt{D}$&-0.0312&-0.0404&-0.2477&-0.0344&-0.0321\\ \hline
\widehat{E}(z)&0.0&0.0&0.0&0.0&0.0\\ \hline

            \end{tabular}
            \caption{Нормальное распределение 1000 элементов}
            \label{tab:norm_1000}
        \end{table}

        \begin{table}[H]
            \centering
            \begin{tabular}{|c|c|c|c|c|c|}
                \hline
                 &$\overline{x}$&$med\ x$&$z_R$&$z_Q$&$z_{tr}$ \\ \hline
$E\left(z\right)$&0.7254&0.0313&3.1868&1.4041&0.2585\\ \hline
$D\left(z\right)$&194.3172&0.3554&4604.129&9.3915&0.3452\\ \hline

            \end{tabular}
            \caption{Распределение Коши 10 элементов}
            \label{tab:cauchy_10}
        \end{table}

        \begin{table}[H]
            \centering
            \begin{tabular}{|c|c|c|c|c|c|}
                \hline
                 &$\overline{x}$&$med\ x$&$z_R$&$z_Q$&$z_{tr}$ \\ \hline
$E\left(z\right)$&0.6253&-0.0102&32.4295&0.0233&0.0094\\ \hline
$D\left(z\right)$&447.9305&0.0246&1107035.055&0.0534&0.0262\\ \hline

            \end{tabular}
            \caption{Распределение Коши 100 элементов}
            \label{tab:cauchy_100}
        \end{table}

        \begin{table}[H]
            \centering
            \begin{tabular}{|c|c|c|c|c|c|}
                \hline
                 &$\overline{x}$&$med\ x$&$z_R$&$z_Q$&$z_{tr}$ \\ \hline
$E\left(z\right)$&-0.2614&0.0&-113.7389&-0.0003&0.0005\\ \hline
$D\left(z\right)$&131.9924&0.0023&29640377.6809&0.0048&0.0023\\ \hline

            \end{tabular}
            \caption{Распределение Коши 1000 элементов}
            \label{tab:cauchy_1000}
        \end{table}

        \begin{table}[H]
            \centering
            \begin{tabular}{|c|c|c|c|c|c|}
                \hline
                 &$\overline{x}$&$med\ x$&$z_R$&$z_Q$&$z_{tr}$ \\ \hline
$E\left(z\right)$&-0.0017&0.0003&0.0012&0.4187&0.3298\\ \hline
$D\left(z\right)$&0.1954&0.1365&0.8186&0.2293&0.1584\\ \hline
$E + \sqrt{D}$&0.4404&0.3699&0.906&0.8975&0.7277\\ \hline
$E - \sqrt{D}$&-0.4437&-0.3692&-0.9035&-0.0602&-0.0682\\ \hline
\widehat{E}(z)&0.19&0.136&0.81&-&-\\ \hline

            \end{tabular}
            \caption{Распределение Лапласа 10 элементов}
            \label{tab:laplace_10}
        \end{table}

        \begin{table}[H]
            \centering
            \begin{tabular}{|c|c|c|c|c|c|}
                \hline
                 &$\overline{x}$&$med\ x$&$z_R$&$z_Q$&$z_{tr}$ \\ \hline
$E\left(z\right)$&-0.0035&-0.0041&0.0086&0.0184&0.0261\\ \hline
$D\left(z\right)$&0.02&0.0121&0.7719&0.0193&0.0128\\ \hline
$E + \sqrt{D}$&0.1377&0.1059&0.8872&0.1573&0.1394\\ \hline
$E - \sqrt{D}$&-0.1448&-0.1141&-0.87&-0.1205&-0.0871\\ \hline
\widehat{E}(z)&0.0&0.0&0.7&0.0&-\\ \hline

            \end{tabular}
            \caption{Распределение Лапласа 100 элементов}
            \label{tab:laplace_100}
        \end{table}

        \begin{table}[H]
            \centering
            \begin{tabular}{|c|c|c|c|c|c|}
                \hline
                 &$\overline{x}$&$med\ x$&$z_R$&$z_Q$&$z_{tr}$ \\ \hline
$E\left(z\right)$&0.0023&0.0017&-0.0219&0.0039&0.0035\\ \hline
$D\left(z\right)$&0.0018&0.001&0.8525&0.0018&0.0011\\ \hline

            \end{tabular}
            \caption{Распределение Лапласа 1000 элементов}
            \label{tab:laplace_1000}
        \end{table}

        \begin{table}[H]
            \centering
            \begin{tabular}{|c|c|c|c|c|c|}
                \hline
                 &$\overline{x}$&$med\ x$&$z_R$&$z_Q$&$z_{tr}$ \\ \hline
$E\left(z\right)$&9.9637&9.834&10.2755&10.895&8.5507\\ \hline
$D\left(z\right)$&1.0235&1.4704&1.9523&1.368&0.8573\\ \hline

            \end{tabular}
            \caption{Распределение Пуассона 10 элементов}
            \label{tab:poisson_10}
        \end{table}

        \begin{table}[H]
            \centering
            \begin{tabular}{|c|c|c|c|c|c|}
                \hline
                 &$\overline{x}$&$med\ x$&$z_R$&$z_Q$&$z_{tr}$ \\ \hline
$E\left(z\right)$&9.997&9.837&10.977&9.9555&9.6981\\ \hline
$D\left(z\right)$&0.0971&0.2039&1.0005&0.1503&0.1158\\ \hline

            \end{tabular}
            \caption{Распределение Пуассона 100 элементов}
            \label{tab:poisson_100}
        \end{table}

        \begin{table}[H]
            \centering
            \begin{tabular}{|c|c|c|c|c|c|}
                \hline
                 &$\overline{x}$&$med\ x$&$z_R$&$z_Q$&$z_{tr}$ \\ \hline
$E\left(z\right)$&10.0033&9.996&11.634&9.998&9.8458\\ \hline
$D\left(z\right)$&0.0098&0.004&0.6395&0.0015&0.0108\\ \hline

            \end{tabular}
            \caption{Распределение Пуассона 1000 элементов}
            \label{tab:poisson_1000}
        \end{table}

        \begin{table}[H]
            \centering
            \begin{tabular}{|c|c|c|c|c|c|}
                \hline
                 &$\overline{x}$&$med\ x$&$z_R$&$z_Q$&$z_{tr}$ \\ \hline
$E\left(z\right)$&-0.0178&-0.0205&-0.0079&0.2859&0.1164\\ \hline
$D\left(z\right)$&0.0985&0.2222&0.0469&0.1304&0.1205\\ \hline

            \end{tabular}
            \caption{Равномерное распределение 10 элементов}
            \label{tab:uniform_10}
        \end{table}

        \begin{table}[H]
            \centering
            \begin{tabular}{|c|c|c|c|c|c|}
                \hline
                 &$\overline{x}$&$med\ x$&$z_R$&$z_Q$&$z_{tr}$ \\ \hline
$E\left(z\right)$&-0.0061&-0.0104&-0.0012&0.0113&0.0085\\ \hline
$D\left(z\right)$&0.0092&0.0279&0.0006&0.0141&0.0176\\ \hline

            \end{tabular}
            \caption{Равномерное распределение 100 элементов}
            \label{tab:uniform_100}
        \end{table}

        \begin{table}[H]
            \centering
            \begin{tabular}{|c|c|c|c|c|c|}
                \hline
                 &$\overline{x}$&$med\ x$&$z_R$&$z_Q$&$z_{tr}$ \\ \hline
$E\left(z\right)$&0.002&0.0028&0.0&0.0038&0.0048\\ \hline
$D\left(z\right)$&0.001&0.0029&0.0&0.0015&0.0019\\ \hline

            \end{tabular}
            \caption{Равномерное распределение 1000 элементов}
            \label{tab:uniform_1000}
        \end{table}

    \section{Обсуждение}
        Можно заметить, что дисперсия характеристик рассеяния для распределения Коши является аномально болишим по сравнению с другими распределениями.
        Даже при увиличении размера выборки ситуация не меняется. Это можно обяснить выбросами, которые наблюдались в результатах предыдущей работы.
    \section*{Примечание}
        С кодом работы и отчета можно ознакомиться по ссылке:\;\url{https://github.com/sqrtyyy/MathStat/tree/master/lab_2}
\end{document}