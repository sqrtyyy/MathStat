\documentclass[a4paper]{article}
\usepackage{amsmath}
\addtolength{\hoffset}{-2.25cm}
\addtolength{\textwidth}{4.5cm}
\addtolength{\voffset}{-3.25cm}
\addtolength{\textheight}{5cm}
\setlength{\parindent}{15pt}

\usepackage[unicode=true, colorlinks=false, hidelinks]{hyperref}
\usepackage[utf8]{inputenc}
\usepackage[english, russian]{babel}
\usepackage{mathtext}
\usepackage[T2A, TS1]{fontenc}
\usepackage{microtype} % Slightly tweak font spacing for aesthetics
\usepackage{amsthm, amssymb, amsmath, amsfonts, nccmath}
\usepackage{nicefrac}
\usepackage{epstopdf}
\usepackage[export]{adjustbox}
\usepackage{float} % Improved interface for floating objects
\usepackage{graphicx, multicol} % Enhanced support for graphics
\usepackage{pdfrender,xcolor}
\usepackage{breqn}
\usepackage{mathtools}
\usepackage{titling}
\usepackage{bm}
\usepackage{centernot}
\usepackage[cal=boondoxo,calscaled=.96]{mathalpha}
\usepackage{marvosym, wasysym} % More symbols
\usepackage{rotating} % Rotation tools
\usepackage{censor} % Facilities for controlling restricted text

\usepackage{fancyhdr}
\pagestyle{fancy}
\fancyhead{}\renewcommand{\headrulewidth}{0pt}
\fancyfoot[L]{}
\fancyhead{}
\fancyfoot{}
\fancyfoot[R]{\thepage}
\begin{document}
    \begin{titlepage}
   \begin{center}
       \vspace*{3cm}
       \large{САНКТ-ПЕТЕРБУРГСКИЙ ПОЛИТЕХНИЧЕСКИЙ УНИВЕРСИТЕТ}
       \vspace{0.4 cm}

       \large\textbf{Институт прикладной математики и механики}
       \vspace{0.4 cm}

       \large{Высшая школа прикладной математики и вычислительной физики}

       \vspace{3 cm}
       \normalsize\textbf{Отчет\\ по лабораторной работе №2\\ по дисциплине\\
«Математическая статистика»}
       \vfill
       \begin{flushright}
            \normalsize{Выполнил студент:\\
            Антонов Алексей\\
            группа: 3630102/80201}
            \vskip\medskipamount
            \normalsize{Проверил:

            к.ф.-м.н., доцент\\
            Баженов Александр Николаевич
            }
       \end{flushright}

       \vspace{0.8cm}


       \normalsize{Санкт-Петербург\\2021 г.}

   \end{center}
\end{titlepage}
    \tableofcontents
    \newpage
    \listoffigures
    \newpage
    \section{Постановка задачи}
    Требуется произвести калибровку шкалы измерителя, с нерегулярными отсчетами по предоставленным данным.\\
    Для этого нужно:
\begin{enumerate}
    \item Определить амплитуду гармонического сигнала по набору отсчетов
    \item Использовать интервальный подход к решению переопределенных СЛАУ для точного определения амплитуды
    \\item Определить фазы отсчетов по амплитуде
\end{enumerate}

\section{Теория}
    \subsection{Интервал неопределенности}
        Мы говорим, что измерение $y = f(t)$ имеет интервалом неопределенности $[a, b], a, b \in \mathbb{R}$, если предполагаем, что это измерение принадлежит какой-то точке этого интервала. Например, если прибор по
        измерению показал значение $\overline{y}$ и известно, что прибор имеет погрешность $\epsilon$, тогда на практике для
        описания неопределенности выбирают интервал $[\overline{y} - \epsilon, \overline{y} + \epsilon]$. Для интервалов в литературе определена
        арифметика.\cite{Bazhenov}
    \subsection{Алгебра интервалов}
        Для интервалов существует стандартная алгебра:
        \begin{itemize}
            \item $\overline{A}+\underline{A}=[\overline{a},\overline{b}]+[\underline{a},\underline{b}]=[\overline{a}+\underline{a},\overline{b}+\underline{b}]$
            \item $\overline{A}-\underline{A}=[\overline{a},\overline{b}]-[\underline{a},\underline{b}]=[\overline{a}-\underline{a},\overline{b}-\underline{b}]$
            \item $\overline{A}*\underline{A}=[\overline{a},\overline{b}]*[\underline{a},\underline{b}]=[\min(\overline{a}\underline{a},\overline{a}\underline{b},\overline{b}\underline{a},\overline{b}\underline{b}),\max(\overline{a}\underline{a},\overline{a}\underline{b},\overline{b}\underline{a},\overline{b}\underline{b})]$
            \item $\overline{A}/\underline{A}=[\overline{a},\overline{b}]/[\underline{a},\underline{b}]=[\overline{a},\overline{b}]*[1 / \underline{a},1 / \underline{b}]$
        \end{itemize}
        \subsection{Интервальная матрица}
            \begin{equation}
                A=
                    \begin{pmatrix}
                        a_{11} & \dots & a_{1n}\\
                        \vdots & \ddots & \vdots\\
                        a_{m1} & \dots & a_{mn}\\
                    \end{pmatrix}\label{eq:IMATRIX}
            \end{equation}
            где $a_{ij}$ - интервал, $i=\overline{1,m},j=\overline{1,n}$.
        \subsection{Интервальная СЛАУ(ИСЛАУ)}
        \begin{equation}
            \left\{
            \begin{array}{ll}
                a_{11}x_1+\ldots+a_{1n}x_n=b_1\\
                \ldots\\
                a_{m1}x_1+\ldots+a_{mn}x_n=b_m\\
            \end{array}
            \right.\label{eq:ISLAU}
        \end{equation}
        где $a_{ij},b_i$ - интервалы, $i=\overline{1,m},j=\overline{1,n}$, или $Ax=B$, где $A=(a_{ij})$ - интервальная матрица, $B=(b_i)$ - интервальный вектор.
    \subsection{Множество решений ИСЛАУ}
    Объединенное множество $\mathsf{\Xi}_{uni}(A,B)=\{x\in{\mathbb{R}^n}|\exists{A'}\in{A},\exists{B'}\in{B},A'x=B'\}$ \\
    Допусковое множество $\mathsf{\Xi}_{tol}(A,B)=\{x\in{\mathbb{R}^n}|\forall{A'}\in{A},\exists{B'}\in{B},A'x=B'\}$ \\\\
    При этом $\mathsf{\Xi}_{tol}(A,B)\subseteq{\mathsf{\Xi}_{uni}(A,B)}$.

\subsection{Задача восстановления зависимости}
    Задача восстановления зависимости – это распространённая задача, в которой по эмпирическим данным требуется построить
    зависимость заданного вида. В реальных задачах восстановления зависимости эмпирические данные почти всегда не точные,
    так как на результаты измерений влияют внешние неконтролируемые факторы и измерительные приборы всегда имеют погрешность
    в измерениях. Возникшую неопределенность входных данных, будем описывать спомощью интервального анализа.
    Пусть $A$ - интервальная матрица, $B$ - интервальный столбец эмпирических данных.\\\\
    Тогда $Ax=B$ - ИСЛАУ, где $x_1,\dots,x_n$ - оценки исходных параметров.\\\\
    Решение данного ИСЛАУ в общем случае представляет собой множество $\mathsf{\Xi}_{uni}(A,B)$.
    Если требуется сильное  согласование параметров с интервальными экспериментальными данными, то решением является
    множество $\mathsf{\Xi}_{tol}(A,B)$.

    \subsection{Определение параметров гармонического сигнала}
Необходимо выполнить масштабирование исходной выборки $\{y_i\}$ - амплитудные значения сигнала в промежуток $[0,1]$ и вычислить амплитуду арксинуса как пересечение прямых, проходящих через линейно зависимые точки. При этом учитываем, что амплитудные значения $y_i$ даны с погрешностью.
    \subsection{Алгоритм поиска коэффициентов прямой $y=a^+x+b^+$ с положительным наклоном}

\begin{enumerate}
    \item Находим множество индексов $I^+=\{I^+_{k0},...,I^+_{kn}\}$, где $I^+_{ki}$ - множество точек, лежащих на одной прямой с положительным наклоном.
    \item При этом каждая точка должна удовлетворять условию $y=a^+i+b^++dk$, где $d$ - смещение из-за периода.
    \item Построить интервальные оценки по спецификации прибора для $i:[i-\frac{1}{2},i+\frac{1}{2}]$, для $y:[y-0.015|y|,y+0.015|y|]$
    \item Составить ИСЛАУ, подставив соответствующие интервальные оценки в СЛАУ вида:
        \begin{equation}
            \begin{pmatrix}
                i_0 & 1 & 0\\
                ... & ... & ...\\
                i_j & 1 & k\\
                ... & ... & ...\\
                i_n & 1 & l\\        
            \end{pmatrix}
            \begin{pmatrix}
                a\\
                b\\
                d\\
            \end{pmatrix}
            =
            \begin{pmatrix}
                y_0\\
                ...\\
                y_j\\
                ...\\
                y_n\\        
            \end{pmatrix}
        \end{equation}
        где $i_j\in{I^+_{kj}}$
    \item Применить метод максимального согласования для нахождения оценок параметров $a$ и $b$
    \end{enumerate}

Таким образом, с помощью найденных $a^+,b^+,a^-,b^-$ находится амплитуда арксинуса как ордината точки пересечения соответствующих прямых.\cite{VKR}

\subsection{Определение фаз отсчетов сигнала}
    Необходимо провести масштабирование $\{y_i\}$ так, чтобы амплитуда стала равной $\frac{\pi}{2}$. Тогда $\Delta{t_i}=\frac{\Delta{y_i}}{2\pi\vartheta}$.
    При этом временной интервал вычисляется для точек, по котором производилось построение прямых, а временной интервал между соседними измерениями
    $\Delta{t_i}$ вычисляется как среднее по всем сигналам.
\section{Реализация}
\begin{itemize}
    \item numpy
    \item matplotlib
    \item math
    \item tolsolvty
\end{itemize}

\section{Результаты}
    Разобьем работу на 2 этапа:
    \begin{itemize}
        \item Амплитудная калибровка
        \begin{itemize}
            \item Подать поочередно сигналы констант и построить по ним усредненные значения для каждой ячейки (набор уровней)
            \item Построить кусочно-линейную интерполяцию сигнала
        \end{itemize}
        \item Применение интервального анализа к временной калибровке
            \begin{itemize}
                \item Определить параметры гармонического сигнала по амплитудным значениям
                \item Определить фазы отсчетов сигнала
            \end{itemize}
    \end{itemize}

    \subsection{Амплитудная калибровка}
        \begin{figure}[H]
            \centering
            \includegraphics[scale=0.8]{src/AmplitudeValues}
            \caption{Амплитуды входного сигнала при измерении константных сигналов, цветом обозначены разные значения констант}
        \end{figure}

    \begin{figure}[H]
        \centering
        \includegraphics[scale=0.8]{src/AmplitudeValuesWithSignals}
        \caption{Оцифрованный сигнал с иллюстрацией константных сигналов}
    \end{figure}

    \begin{figure}[H]
            \centering
            \includegraphics[scale=0.8]{src/Regress}
            \caption{Пример вычисления регрессии}
        \end{figure}

    \begin{figure}[H]
        \centering
        \includegraphics[scale=0.8]{src/InterpolatedSignal}
        \caption{Интерполированный сигнал}
    \end{figure}
    \subsection{Применение интервального анализа к временной калибровке}
        \begin{figure}[H]
            \centering
            \includegraphics[scale=0.8]{src/Signal[0,1]}
            \caption{Масштабированный сигнал на [0,1]}
        \end{figure}
        \begin{figure}[H]
                \centering
                \includegraphics[scale=0.8]{src/ScaledSignal}
            \caption{Арксинус сигнала}
        \end{figure}

        \begin{figure}[H]
            \centering
            \includegraphics[scale=0.8]{src/ArcsinParams}
            \caption{Нахождение амплитуды сигнала}
        \end{figure}

        \begin{figure}[H]
            \centering
            \includegraphics[scale=0.8]{src/TimePeriods}
            \caption{Временная шкала сигнала}
        \end{figure}

        \begin{figure}[H]
            \centering
            \includegraphics[scale=0.8]{src/TimeHistogram}
            \caption{Гистограмма распределения ширин временных бинов}
        \end{figure}

\section{Обсуждение}
    \begin{itemize}
        \item Использование функции арксинуса, позволяет решить проблему недостаточности точек, так как появится возможность использовать
              практически все точки, а не только в линейной области.
        \item Использование интервального анализа для поиска коэффициента прямых позволяет найти коэффициенты прямых в
              условиях, когда амплитудные значения даны с погрешностью, а также из-за неравномерных временных отсчетов,
              возникает погрешность в индексации.
        \item На гистограмме распределения временных ширин наблюдается бимодальность, что соответствует
              измерениям других авторов. \cite{VKR}.
    \end{itemize}

    \section{Примечание}
        С кодом работы и отчета можно ознакомиться по ссылке:\;\url{https://github.com/sqrtyyy/MathStat/tree/master/coursework}

\section{Список используемой литературы}
    \begin{thebibliography}{3}
        \bibitem{Bazhenov}
        Баженов А.Н. Интервальный анализ. Основы теории и учебные примеры. - Спб., 2020
        \bibitem{VKR}
        Билев Ф.А. Исследование применения интервального подхода к задаче калибровки шкалы измерителя с нерегулярными отсчетами // Бакалаврская работа. / Рук.: Баженов А.Н. - С.-П. : СПБПУ. - 2017.
    \end{thebibliography}

\end{document}